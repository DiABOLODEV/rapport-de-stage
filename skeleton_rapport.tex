\documentclass[a4paper, 12pt, onecolumn, openright, oneside]{report}

 \usepackage{setspace} 
\usepackage[ansinew]{inputenc} 
\usepackage[T1]{fontenc} 
\usepackage[francais]{babel} 
\usepackage{color}
\usepackage{enumitem}
\usepackage{graphicx}
\usepackage{fancybox}
\usepackage{color}
\definecolor{rouge}{RGB}{255,112,119}
\definecolor{rouge-clair}{RGB}{255,163,168}
\definecolor{rouge-tres-clair}{RGB}{255,217,219}

\title{Rapport de stage}
\author{\textsc{Thibault} - \textsc{Gauthier}}
\date{\today}

\begin{document}
   \maketitle
  
   \chapter*{Remerciements}
 
   \tableofcontents
   
   \chapter*{Introduction}
   
   \chapter{La soci�t� Capgemini}
      \section{Historique}
      \section{Le site de Rennes}
      \section{Le groupe TMA OSS}
      \section{Le service SIG}	
      \section{Le projet G�ofibre}
   
   \chapter{Travai r�alis�}
      \section{Introduction}
      \section{Mont�e en comp�tence}
      \section{Tests d'int�gration G1R6}
      \subsection{Zoom sur un test d'int�gration}
      \paragraph{Test de Filtrage} L'ensemble de tests de filtrage est cat�goris� en importance \emph{P3} cela signifie qu'il est secondaire (p1,p2,p3). Nous allons voir ensemble le test \emph{Activation d'un filtre pr�programm� param�trable}. Tout d'abord il faut ouvrir la plateforme QC gr�ce au navigateur Internet Explorer et nos identifiants dans l'entreprise.
      \includegraphics{content/zoom-test-filtrage/connexion-qc.png}
      
      \section{Conception G1R7}
   
   \chapter{Bibliographie / Webographie}
   
   \chapter{R�sum�}
   
   \chapter{Resume}
   
   \chapter*{Conclusion}

   \appendix 
   	    	 \chapter{Travaux par semaine}
   	    \begin{enumerate}[label= Semaine \no\arabic*.]
   	       \setcounter{enumi}{10}
   	       \item Pr�sentation de l'�tage \textsc{TMA OSS} par le directeur de projets \textsc{Arnaud Bellina}. Installation de mon poste de travail. Installation des diff�rents logiciels et lecture de la documentation ``KIT D'ACCUEIL''. Visite autonome du b�timent (CE, caf�taria, infirmerie, support). Cr�ation du badge d'entr�e. Int�gration dans l'�quipe
   	       \item Pr�sentation des outils essentiels pour le d�veloppement par \textsc{Anne-Sophie Lescop}. Importation des projets qui constitue l'application G�ofibre � partir de SVN. Decorticage du code.
   	       \item Pr�sentation SIG par Eric Langlois. Lecture de la documentation ``Sp�cifications fonctionnelles''. Premi�re t�che de d�veloppement : Externalisation de la Minimap. Lecture de la documentation, apprentissage de g�ofibre.
   	       \item Ga�l m'a montr� comment fonctionne l'interface des violations Sonar. Correction d'un bug c�t� applicatif ou une conf en base de donn�es est mal r�cup�r�e.
   	       \item T�che toolbox
   	    
   	       \item Tests d'int�gration sur la R�union
   	           \begin{enumerate}[label = Test set \no\alph*.]
   	           \item \emph{P1} - Gestion infrastructure - Recalage GC (10 tests)
   	           \item \emph{P1} - Gestion infrastructure - Zone de recalage (8 tests)
   	           \item \emph{P1}
   	         \end{enumerate}
   	       \item Tests d'int�gration sur la Guyane
   	           \begin{enumerate}[label = Test \no\alph*.]
   	           \item Gestion infrastructure - Site supports (12 instances de tests)
   	         \end{enumerate}
   	       \item Tests d'int�gration sur la Guadeloupe : \begin{enumerate}[label = Test set \no\alph*.]
   	          \item \emph{P1} - Gestion infrastructure - Site supports 
   	           \item \emph{P3} - Filtrage 
   	           \item \emph{P3} - Gestion des droits 
   	           \item \emph{P3} - G�osignets 
   	           \item \emph{P3} - Localisation objet m�tier 
   	           \item \emph{P1} - Exports - Dossier OPGC - Base arri�re de PM
   	           \end{enumerate}
   	       \item
   	       \item
   	       \item
   	       \item
   	       \item
   	       \item
   	       \item
   	       \item
   	       \item
   	       \item
   	       \item
   	       \item
   	       \item
   	       \item
   	       \item
   	       \item 
   	     \end{enumerate}
   	 \chapter{Annexe 2}
   	 \chapter{Annexe 3}
\end{document}