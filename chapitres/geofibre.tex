\chapter{Le projet Géofibre}
\section{Objet}
Le projet \textit{Geofibre} a pour objet de fournir une application de \textit{SIG} pour ORANGE dans le domaine de \textit{FTTH}\footnote{Fiber To The Home : C'est le réseau trés haut débit de fibre optique pour les clients résidentiels.}.\\
L'application se présente sous la forme d'une page Web permettant de gérer et concevoir des données descriptives du réseau de la fibre optique pour les clients résidentiels en France (\textit{ndlr: Et depuis peu dans les départements d'Outre-Mer}) en temps réel avec plusieurs utilisateurs connectés simultanément.\\
Cette application est principalement destinée aux chargés d'affaire et sous-traitant FTTH.\\
Elle a pour mission, par exemple, de faire évoluer le réseau en permettant la conception sur l'application pour ensuite l'imprimer et l'installer sur le terrain ou par exemple avoir une vision globale des installations sur une commune.\\
En terme de charge, elle comptabilise en 2015 jusque a \textbf{1150 utilisateurs simultanés}.\\
Techniquement Geofibre est basé sur le progiciel ArcGIS de l'éditeur ESRI.

\newpage
\section{Historique}
Le lancement du projet a eu lieu en 2010. Jusqu'en 2012 le projet s'est développé dans les locaux du client dans la ville de Lannion suivant la méthode de gestion de projet \textit{AGILE}.\\
Les employés de la société Capgemini étaient à cette époque présents dans les locaux du client pour travailler en tant qu'assistants technique.
Par la suite le développement du projet s'est réalisé dans les locaux de Capgemini, au Spiréa à Rennes suivant la méthode de gestion de projet \textit{LEAN}

\begin{colbox}{{HTML}{A3E8FF}}{La méthode LEAN\\ }
	\textbf{Objectif} : Améliorer de façon continue la performance en termes de qualité, coûts et délais de livraison.
	\\\textbf{Origine} : Apparue dans le seconde moitié du XXème siècle avec l'entreprise \textit{Toyota}. La production de voiture répond à une demande, ainsi les stocks sont quasi inexistants.
	\\\textbf{Principe} : Créer de la valeur ajoutée pour le client avec un minimum de gaspillage et en livrant un maximum de qualité.
	\\\textbf{En tant que développeur} : Des indicateurs de qualité de code à améliorer au fil des versions, des délais de livraison du service fini à respecter.
\end{colbox}

Depuis ses débuts Géofibre a évolué de manière significative. \'A l'heure actuelle le projet est à sa 7ème version mineure(cf. \ref{versionning})et des évolutions sont prévues, d'autant que le gouvernement Français souhaite développer la fibre optique sur l'ensemble du territoire Français.
