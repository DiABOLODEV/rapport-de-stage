\chapter{Le domaine de compétence SIG}
\section{Qu'est ce que le SIG ?}
Un système d'information géographique (SIG) est un système d'information conçu pour recueillir, stocker, traiter, analyser, gérer et présenter tous les types de données spatiales et géographiques. L’acronyme SIG est parfois utilisé pour définir les « sciences de l’information géographiques » ou « études sur l’information géospatiales ». Cela se réfère aux carrières ou aux métiers qui travaillent avec des systèmes d’information géographique et dans une plus large mesure avec les disciplines de la géo-informatique. Ce que l’on peut observer au-delà du simple concept de SIG a trait aux données de l’infrastructure spatiale.
Dans un sens plus général, le terme de SIG décrit un système d’information qui intègre, stocke, analyse, et affiche l’information géographique. Les applications liées aux SIG sont des outils qui permettent aux utilisateurs de créer des requêtes interactives, d’analyser l’information spatiale, de modifier et d’éditer des données au travers de cartes et d’y répondre cartographiquement. La science de l’information géographique est la science qui sous tend les applications, les concepts et les systèmes géographiques.
Le SIG est un terme général qui se réfère à un certain nombre de technologies, de processus et de méthodes.
\section{\'Equipes et projets}
\subsection{SIGEO}
\subsection{VASCO}
\subsection{Geofibre}
