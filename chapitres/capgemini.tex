\chapter{La société Capgemini}
\section{Présentation}
\section{Histoire}
\section{Géographie}
\subsection{A l'international}
\subsection{En France}
\section{Capgemini à Rennes}
\subsection{Types de services}

\begin{comment}
\section{Les premiéres années (1967-1975)}
Les racines de Capgemini se trouvent é Grenoble, oé Serge Kampf fonda Sogeti en 1967. En 1975, l'acquisition de 2 importantes sociétés de services IT, CAP et Gemini Computer Systems, fait de l'entreprise un leader en Europe, avec une présence dans 21 pays.

\section{Le temps de l'expansion (1975-1989)}
Le Groupe continue de grandir et de se développer ; son action se concentre sur des domaines allant des solutions d'investissement de capital  aux services intellectuels de qualité. En 1989, restructuration interne, expansion européenne et pénétration du marché américain American permettent é Capgemini de faire partie des 5 leaders mondiaux de ce secteur.
Construire le futur (1998 - aujourd'hui)

Au cours des derniéres années, qui se sont révélées particuliérement difficiles pour le secteur des services informatiques, il est apparu nécessaire de ré-équilibrer le portefeuille déactivités du Groupe en faveur de deux de ses métiers : léassistance technique de proximité et léinfogérance. Léacquisition de Transiciel, fin 2003, a permis é Capgemini de doubler la taille de Sogeti, léentité créée en 2001 dans le domaine de léassistance technique. Aujourdéhui Sogeti, qui représente 15% du chiffre déaffaires du Groupe, est le leader européen dans ce segment de marché, tout en se développant aussi aux Etats-Unis.
Nous avons également développé une approche baptisée Rightshoreé, qui s'appuie é la fois sur les capacités offshore  éavec notamment 20 500  collaborateurs basés en Inde en 2009, suite é l'acquisition of Kanbay, ainsi que des centres offshore au Maroc et en Argentine é et sur les ressources nearshore dans des pays incluant la Pologne et l'Espagne.

\section{De nouvelles stratégies de croissance (1990-1997)}
Capgemini construit é un niveau mondial sa pratique du conseil en management, gréce é une série d'acquisitions stratégiques, parmi lesquelles United Research (1990) et le Groupe Mac (1991) aux Etats-Unis. Ces acquisitions ont permis é Capgemini de se développer dans le domaine du conseil en management. Le Groupe a par ailleurs progressivement étendu ses activités en Europe avec notamment léacquisition de Data Logic en Scandinavie, déHoskyns au Royaume-Uni (1990), et de Volmac aux Pays-Bas (1992), complétées par celles de Gruber Titze \& Partners en Allemagne (1993) puis de Bossard en France (1997).
Plus récemment, léacquisition déErnst \&. Young Consulting (2000) a renforcé le profil international du Groupe, en accroissant trés sensiblement sa présence en Amérique du Nord et dans un certain nombre déautres pays déEurope.

\section{Déimportantes signatures dans le domaine de l'é'infogérance}


Dans le domaine de léinfogérance, le Groupe a recueilli en 2004 les fruits des efforts menés pour asseoir sa présence aussi bien en Europe quéen Amérique du Nord. Ainsi le contrat Aspire signé fin 2003 avec léadministration fiscale britannique a démarré tout début 2004.
Lui ont succédé un contrat de 3,5 milliards de dollars sur 10 ans avec le groupe américain TXU (mai 2004) sur le marché de léénergie, puis en novembre 2004 un contrat de 1,6 milliards déeuros sur 10 ans avec le groupe franéais Schneider Electric.

  \section{Profil}
	\paragraph{Marque de fabrique}
  \section{Solutions et services}
	Capgemini a un large pannel de services et solutions dans 5 domaines différents :
	\begin{enumerate}
		\item Consultant
		\item Infrastructure
		\item Application
		\item Externalisation
		\item Services professionnels locaux
	\end{enumerate}

  \section{Positionnement par rapport à la concurrence}
\end{comment}
  %
