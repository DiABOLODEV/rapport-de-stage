\chapter{La société Capgemini}
\subsection*{Introduction}
Capgemini est une société de services spécialisée en génie informatique (ESN\footnote{Entreprise de services du numérique} ou SSII\footnote{Société de services en Ingénierie Informatique}).
\\Elle est présente dans le monde entier et elle est la première dans son domaine en terme de chiffre d'affaire.
\newpage
\section{Fiche d'identité}
\begin{description}
  \item[Raison sociale] : Capgemini
  \item[Année de création] : 1967
  \item[Fondateur] : Serge Kampf
  \item[Forme juridique] : Société anonyme à conseil d'administration
  \item[Siège social] : Paris
  \item[Directeur Génral] : Paul Hermelin
  \item[Activité] : ESN\footnote{Entreprise de Services du Numérique}
  \item[Présence internationale] : 40 pays
  \item[Effectif en 2014] : 145 000
  \item[Principales activités] : Conseil en management, Intégration de systèmes, Infogérance
  \item[Chiffre d'affaire en 2014] : 10,6 milliards d'euros
  \item[Secteurs d'acitvité] : secteur public, services financiers, grande consommation, télécommunications, services universels
\end{description}
\begin{figure}[h]
  \captionbox{Logo de Capgemini\label{fig:dummy}}{
    \includegraphics[width=5cm]{images/logo_capgemini.png}
  }
  \captionbox{Implémentation de Capgemini\label{fig:dummy}}{
    \includegraphics[width=5cm]{images/logo_capgemini.png}
  }
\end{figure}

\section{Métiers et activités}
Capgemini est l'un des leaders mondiaux dans le domaine du conseil, des services informatiques, et de l'infogérance.
Le groupe définit ses métiers en 4 grandes catégories.
\\
\begin{itemize}
\item le conseil en management
\item l'intégration de systèmes et le développement d'applications
\item l’infogérance (Outsourcing Services - OS)
\item l'assistance technique et services de proximité
\end{itemize}
