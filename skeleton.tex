\documentclass[a4paper, 12pt, onecolumn, openright, oneside]{report} % structure du rapport de stage - feuille a4 juste recto - commencer les nouvelles pages a droite
\usepackage{setspace} % afin d ajuster linterligne - ex onehalfspace
\usepackage[ansinew]{inputenc} % accepte les car accentues - [latin1] pour Linux - [ansinew] pour Windows
\usepackage[T1]{fontenc} % guillements a la french
\usepackage[francais]{babel} % langue francaise
%\usepackage{charter}
\usepackage{listings}
% \usepackage[top=2cm, bottom=2cm, left=3cm, right=2cm]{geometry} % TODO marges

%definition des couleurs
%\definecolor{emph}{rgb}{0.1,0.8,1}

% informations de la page de garde
\title{Rapport de stage}
\author{Thibault \bsc{Gauthier}}
\date{27 Avril 2015}


%Utilisation de l'environnement lstlisting  pour citer du code
\lstset{ %
language=C,        % choix du langage
basicstyle=\footnotesize,       % taille de la police du code
numbers=left,                   % placer le num�ro de chaque ligne � gauche (left) 
numbers=right,                  % placer le num�ro de chaque ligne � droite (right)
numberstyle=\normalsize,        % taille de la police des num�ros
numbersep=7pt                  % distance entre le code et sa num�rotation
}

% le rapport commence
\begin{document}
\maketitle % genere la page de garde
\begin{onehalfspace} % interligne 1.5
\part{Commandes \LaTeX{}}
\chapter{listes}
\section{listes � puces}
\begin{itemize}
\item pou
\item puce
\item lente
\end{itemize}
\section{liste num�rot�es}
\begin{enumerate}
\item best pou
\item trou du pou
\item pou-cette
\end{enumerate}
\section{liste de description}
\paragraph{utilit�} Pour l'explication des acronymes ou de mots sp�cifiques
\begin{description}
\item[SIG :] Syst�me d'Information G�ographique
\item[TMA :] Maintenance Applicative
\item[OSS :] Orange services
\end{description}
\section{Mise en forme du texte}
\fbox{un lapin}
\texttt{un lapin}
\textsl{un lapin}
\textbf{un lapin}
%\textcolor{emph}{texte en couleur}
Pour emphaser un \emph{texte}  utiliser emph
\section{Citations}
\begin{quote}
Tout individu a droit � la vie, � la libert� et � la s�ret�de sa personne.
 \end{quote}
 \verb| Mon bout de code {} |
\begin{lstlisting}
if (age == 2)
{
  printf("Salut bebe !");
}
else if (age == 6)
{
  printf("Salut gamin !");
}
else if (age == 12)
{
  printf("Salut jeune !");
}
\end{lstlisting}
\chapter*{Conclusion}
\end{onehalfspace}
\end{document}


