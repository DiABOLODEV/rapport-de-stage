\chapter{Carnet de bord des travaux réalisés par semaine}
\begin{enumerate}[label= Semaine \no\textbf{\arabic*.},itemsep=20pt]
\setcounter{enumi}{10}

\item \textbf{Visite, présentation et rencontre} avec les équipes de la ferme d'applications \textsc{TMA OSS\footnote{Tierce Maintenant Applicative des applications orientés réseau d'Orange}}. Explication de l'activité par le chef de service \textsc{Arnaud Bellina}.
\newline Visite, présentation et rencontre avec les différents services du bétiment de Capgemini (Infirmerie, CE, Cafétaria, RH, Assistante) .
\newline \textbf{Installation de mon poste de travail} au sein de l'openspace de l'équipe Géofibre et intégration supervisée par le chef de groupe \textsc{Patrick Veillon} et la chef de projet \textsc{Anne-Sophie Lescop}.
\newline \textbf{Installation des logiciels} et \textbf{lecture} de la documentation ainsi que du code qui compose le projet Géofibre épaulé par l'équipe.

\item \textbf{Montée en compétence} générale sur l'application Géofibre.
\item \textbf{Développement de la version G1R6 Front (IHM Flex)} Externalisation des systémes de projection, emprise, échelles, minimap

\item \textbf{Développement de la version G1R6 Back (Serveur, Toolbox)} Vérification de la gestion de la projection

\item \textbf{Développement de la version G1R6 Back (Serveur, Toolbox)} Aiguillage servlet
\newline \textbf{Développement de la version G1R6 Back (Serveur, Toolbox)} Impact code appelant

\item \textbf{Tests d'intégration de la version G1R6 sur la Réunion}
\begin{enumerate}[label = Tests \no\arabic*.,align=left]
\item \emph{\colorbox{rouge}{P1}} - Gestion infrastructure - Recalage GC
\item \emph{\colorbox{rouge}{P1}} - Gestion infrastructure - Zone de recalage
\item \emph{\colorbox{rouge-clair}{P2}} - Exploitation - Import RCV (Référenciel Commune Voies)
\item \emph{\colorbox{rouge-clair}{P2}} - Localisation adresse
\item \emph{\colorbox{rouge-tres-clair}{P3}} - Purge des fichiers (multi instance)
\end{enumerate}

\item \textbf{Tests d'intégration de la version G1R6 sur la Guyane}
\begin{enumerate}[label = Tests \no\arabic*.,align=left]
\item  \emph{\colorbox{rouge}{P1}} - Gestion FTTH - Cébles
\item \emph{\colorbox{rouge}{P1}} - Gestion FTTH - Parcours
\item\emph{\colorbox{rouge}{P1}} -  Gestion FTTH - Zone de travail
\item \emph{\colorbox{rouge}{P1}} - Gestion infrastructure - Itinéraires GC
\item \emph{\colorbox{rouge}{P1}} - Gestion infrastructure - Site supports
\item \emph{\colorbox{rouge-tres-clair}{P3}} - Filtrage
\item \emph{\colorbox{rouge-tres-clair}{P3}} - Gestion des droits
\item \emph{\colorbox{rouge-tres-clair}{P3}} - Géosignets
\item \emph{\colorbox{rouge-tres-clair}{P3}} - Outil de mesure
\item \emph{\colorbox{rouge-tres-clair}{P3}} - Sauvegarde du contexte
\item \emph{\colorbox{rouge-tres-clair}{P3}} - Table des matiéres
\end{enumerate}

\item \textbf{Tests d'intégration de la version G1R6 sur la Guadeloupe}
\begin{enumerate}[label = Tests \no\arabic*.,align=left]
\item \emph{\colorbox{rouge}{P1}} - Gestion infrastructure - Site supports
\item \emph{\colorbox{rouge}{P1}} - Exports - Dossier OPGC - Base arriére de PM
\item \emph{\colorbox{rouge-clair}{P2}} - Méj adresse des immeubles depuis optimum
\item \emph{\colorbox{rouge-clair}{P2}} - Exploitation - majBatchData
\item \emph{\colorbox{rouge-clair}{P2}} - Statistiques
\item \emph{\colorbox{rouge-tres-clair}{P3}} - Filtrage
\item \emph{\colorbox{rouge-tres-clair}{P3}} - Gestion des droits
\item \emph{\colorbox{rouge-tres-clair}{P3}} - Géosignets
\item \emph{\colorbox{rouge-tres-clair}{P3}} - Localisation objet métier
\end{enumerate}

\item \textbf{Tests d'intégration de la version G1R6 sur la Martinique}
\begin{enumerate}[label = Tests \no\arabic*.,align=left]
\item  \emph{\colorbox{rouge}{P1}} - Gestion FTTH - Cébles
\item \emph{\colorbox{rouge}{P1}} - Gestion FTTH - Parcours
\item \emph{\colorbox{rouge}{P1}} - Gestion infrastructure - Itinéraires GC
\item \emph{\colorbox{rouge}{P1}} - Gestion FTTH - Projets
\item \emph{\colorbox{rouge}{P1}} - Gestion FTTH - Schéma directeur
\item \emph{\colorbox{rouge}{P1}} - Gestion FTTH - Régles d'ingienerie
\item \emph{\colorbox{rouge}{P1}} - Décalages horaires
\item \emph{\colorbox{rouge-clair}{P2}} - Statistiques
\item \emph{\colorbox{rouge-tres-clair}{P3}} - Filtrage
\item \emph{\colorbox{rouge-tres-clair}{P3}} - Gestion des droits
\item \emph{\colorbox{rouge-tres-clair}{P3}} - Outil de mesure
\item \emph{\colorbox{rouge-tres-clair}{P3}} - Sauvegarde du contexte
\end{enumerate}
\textbf{Prise en main du logiciel ArcMap de la suite ArcGis.}
\item \textbf{Tests de non-regression de la version G1R6 sur la France métropolitaine}
\begin{enumerate}[label = Tests \no\arabic*.,align=left]
\item  \emph{\colorbox{rouge}{P1}} - Impression
\end{enumerate}
\textbf{Anomalie relevé sur les zones d'égilibilités}
\newline
\textbf{Formation E-Learning}
\begin{enumerate}[label = Formation \no\arabic*.,align=left]
	\item \textit{Utiliser efficacement l'email et la messagerie isntantanee}
	\item \textit{Utiliser le Brown Paper}
	\item \textit{Utiliser du Portail MyLearning}
\end{enumerate}
\item \textbf{Tests de non-regression de la version G1R6 sur la France métropolitaine}
\begin{enumerate}[start = 2,label = Tests \no\arabic*.,align=left]
\item  \emph{\colorbox{rouge}{P1}} - Gestion FTTH - Cébles
\item  \emph{\colorbox{rouge}{P1}} - Gestion FTTH - Régles d'ingienerie
\end{enumerate}
\textbf{Formation E-Learning}
\begin{enumerate}[label = Formation \no\arabic*.,align=left]
	\item \textit{Les fondamentaux du test logiciel}
\end{enumerate}

\newpage
\item \textbf{Présentation du déroulement de mon stage} Collecte d'informations sur le centre de service  \textsc{TMA OSS\footnote{Tierce Maintenant Applicative des applications orientés réseau d'Orange}} et le domaine de compétence \textsc{SIG\footnote{Systéme d'Information Géographique}} auquel se rattache le projet Géofibre sur lequel j'effectue mon stage. Création d'un diaporama pour cette présentation. Réalisation de la présentation avec une dizaine de stagiaires, la responsable DRH et les différents chefs de projets.
\newline
\textbf{Formation E-Learning}
\begin{enumerate}[label = Formation \no\arabic*.,align=left]
	\item \textit{La politique anti-corruption du groupe}
	\item \textit{Les lois de la concurrence}
	\item \textit{Les normes écologique du groupe}
	\item \textit{Le code éthique dans la relation client}
\end{enumerate}
\item
\textbf{Correction d'anomalies hors-garantie éligibles pour la version G1R7}
\textbf{Redaction et passage des \textsc{TU\footnote{Tests Unitaires}} relatifs aux corrections}
\begin{enumerate}[label = Correction \no\arabic*.,align=left]
	\item \textit{Repositionnement d'immeubles en masse} - Perte de la sélection d'immeubles aprés avoir annulé une fenétre de choix d'immeuble.
	\item \textit{Repositionnement d'immeubles séquentiel} - Perte de la sélection d'immeubles aprés avoir annulé une fenétre de choix d'immeuble.
	\item \textit{Visu Shape} - Message d'erreur a tord "Le nombre maximum de fichiers visualisés simultanément est de 5".
\end{enumerate}
\textbf{Formation E-Learning}
\begin{enumerate}[label = Formation \no\arabic*.,align=left]
	\item \textit{Communiquer avec assurance}
	\item \textit{Entretenir de bons rapports avec le client}
\end{enumerate}

\item
\textbf{Correction d'anomalies hors-garantie éligibles pour la version G1R7}
\textbf{Redaction et passage des \textsc{TU\footnote{Tests Unitaires}} relatifs aux corrections}
\begin{enumerate}[label = Correction \no\arabic*.,align=left]
	\item \textit{Sites supports} - Perte d'information du champs gestionnaire lors de la duplication si celui-ci a la valeur "39" en production. (en attente d'informations d'Orange)
	\item \textit{Cébles, alvéoles} -  Suppression des données d'alvéoles non homogéne (en attente d'informations d'Orange)
\end{enumerate}
\textbf{Formation E-Learning}
\begin{enumerate}[label = Formation \no\arabic*.,align=left]
	\item \textit{Eléments déune équipe soudée}
	\item \textit{Etablir des relations de confiance}
	\item \textit{Etre un membre efficace au sein déune équipe}
\end{enumerate}
\textbf{Breizhcamp}
\item
\textbf{Correction d'anomalies hors-garantie éligibles pour la version G1R7}
\textbf{Redaction et passage des \textsc{TU\footnote{Tests Unitaires}} relatifs aux corrections}
\begin{enumerate}[label = Correction \no\arabic*.,align=left]
	\item \textit{Connexion} -  Geofibre ne gére pas la casse du cu\_id d'un utilisateur.
	\item \textit{Impression Libre et Casage} - Perte de la valeur par défaut du champ Résolution
\end{enumerate}
\textbf{Formation E-Learning}
\begin{enumerate}[label = Formation \no\arabic*.,align=left]
	\item \textit{Limitation des voleurs de temps}
	\item \textit{Contréler son stress}
	\item \textit{Planifier et hierarchiser son temps}
\end{enumerate}
\item
\textbf{Support é Taher qui viens d'arriver sur le projet}\\
\textbf{Correction d'anomalies hors-garantie éligibles pour la version G1R7}\\
\textbf{Redaction et passage des \textsc{TU\footnote{Tests Unitaires}} relatifs aux corrections}\\
\begin{enumerate}[label = Correction \no\arabic*.,align=left]
	\item \emph{\colorbox{rouge}{majeure}} \textit{Flux cables} -  Echec de l'import sur présence de point virgule , simple guillement ou double guillemet
\end{enumerate}
\textbf{Détection de la version d'anomalie}\\
Certaines anomalies sont en garantie (versions G1R4, G1R5, G1R6) dans quel cas si le client les trouve il faudra les corriger.
D'autres sont hors-garantie (< G1R4) Dans ce cas il faut les annoncer aux clients et ils décident si ils veulent les corriger ou non.
\\
Pour cela il faut détecter ou est-ce que l'anomalie est située dans le code et voir é quel moment les changements ont été commité sur le gestionnaire de version SVN. En fonction de la date du commit ou du TAG
ont peut remonter au numéro de version.
\begin{enumerate}[label = Détection de la version d'anomalie \no\arabic*.,align=left]
	\item \textit{Points techniques} - Il est possible créer un PT avec une référence de plus de 25 caractéres
	\item \textit{Points techniques} - Import - Probléme d'encodage dans les comptes rendus
\end{enumerate}
\textbf{Correction d'anomalies hors-garantie éligibles pour la version G1R7}\\
\textbf{Redaction et passage des \textsc{TU\footnote{Tests Unitaires}} relatifs aux corrections}
\begin{enumerate}[label = Correction \no\arabic*.,align=left]
	\item \textit{Recalcul nombre d'EL} -majBatchData.ksh,KO si  la zone est é cheval sur deux communes [ksh, SQL, PostGIS]
\end{enumerate}
\textbf{Les spécifications de la version G1R7 ont été livré et validé. Les développements peuvent commencer !}
\item
\textbf{G1R7} Lecture assidu des spécifications\\
\textbf{Correction d'anomalies hors-garantie éligibles pour la version G1R7}\\
\textbf{Redaction et passage des \textsc{TU\footnote{Tests Unitaires}} relatifs aux corrections}
\begin{enumerate}[label = Correction \no\arabic*.,align=left]
	\item \textit{Visu Shape} -sauvegarde dans le contexte utilisateur d'un shape non valide [IHM]
\end{enumerate}
\textbf{Développement G1R7 (développement, écriture et passage des tests unitaires)}
\begin{enumerate}[label = Développement \no\arabic*.,align=left]
	\item [Site support] Ajout du champs déployeur en BDD
	\item [Site support] Ajout du champs déployeur dans l'IHM
\end{enumerate}
\item
\textbf{Développement G1R7 (développement, écriture et passage des tests unitaires)}
\begin{enumerate}[label = Développement \no\arabic*.,align=left]
	\item [Annexe C3A] Nouvelle gestion du diamétre des parcours
	\item [Publication de schéma directeur] Choix Opérateur IHM autres impacts
	\item [Publication de schéma directeur] Modification des vues d'extraction
	\item [Publication de schéma directeur] Extractions filtrées sur champ Opérateur
	\item [Publication de schéma directeur] Impressions filtrées sur champ Opérateur
\end{enumerate}


\item
\item
\item
\item
\item
\item
\item
\item
\end{enumerate}
