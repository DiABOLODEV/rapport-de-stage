%\SetLabelAlign{margin}{\llap{#1~~}} %label description dans la amrge

\begin{titlepage}
  %  \begin{minipage}[t]{0.5\linewidth} %t = top%
  %  \flushleft {\large \textbf{CV de Thibault Gauthier}}
  %1  \end{minipage}
  %  \hfill

  %\begin{minipage}[t]{0.5\linewidth}
  %  \flushright \includegraphics[width = 3cm]{images/thibaultgauthier.jpg}

  \tikzoverlay[text width=2.2cm] at (15.5cm,-0.3cm) {
    \tikz node (label) at (0,0)[]{
        \includegraphics[width=2.2cm]{images/thibaultgauthier.png}
    };
};
  {\large \textbf{Thibault Gauthier - Développeur informatique junior}}
  \\\hr\\[1cm]
  %col 1
  \begin{minipage}[t]{8.7cm}

    \framebox[8.7cm]{\textbf{Formations}}\\\\
    \textbf{2013-2015} \\
    \textsc{Master MIAGE\footnotemark option SIAD\footnotemark}
    \\\textit{ISTIC de Rennes (35)}
    \\\\Bases de données décisionnelles / objet, contrôle de gestion,
    coopération et concurrence dans les systèmes et réseaux,
    j2ee, recherche opérationnelle,management de
    projet et intégration d'applications, prospective et
    marketing, entrepôts de données, veille technologique
    \\\\\textbf{2012-2013}\\
    \textsc{Licence MIAGE}
    \\\textit{ISTIC de Rennes (35)}
    \\\\Algorithme des graphes, organisation des systèmes d'exploitation,
    bases de données, bases de modélisation par objets, programmation, comptabilité de gestion,
    environnement juridique, système, filles d'attente et
    gestion de stock, analyse de données, théorie des
    langages, organisation des entreprises, gestion financière,
    techniques d'expression
    \\\\\textbf{2010-2012} \\
    \textsc{DUT Informatique}
    \\\textit{ISTIC de Rennes (35)}
    \\\\Méthodologie de la programmation, apprentissage de
    langages (C, C++, Java, HTML, Python), conception de
    systèmes d'informations, modélisation Merise et UML,
    création et interrogation de bases de données (Oracle,
    MySQL), interface dynamique avec logiciels web (PHP,
    HTML), utilisation, programmation et administration des
    Unix et Windows, architecture des ordinateurs,
    protocoles réseaux et application Internet
    \\\\\textbf{2008 - 2010} \\
    \textsc{Bac S option SI\footnotemark}
  \end{minipage}
  \hfill
  %col 2
  \begin{minipage}[t]{8.7cm}

    \framebox[8.7cm]{\textbf{Expériences professionnelles}}\\\\
    \textbf{2015 - Stage de 6 mois} \\
    \textsc{Capgemini} -
    \textit{ESN à Rennes (35)}
    \\\\Soutient, maintenance et développement d'évolutions sur l'application \textit{Geofibre} qui permet de visualiser et gérer le réseau FTTH d'\textit{Orange} en France.
    \\\\\textit{SIG, Arcgis, API Rest, Flex, Flash, Eclipse, FlashBuilder, shell, Linux, QC, Tests, PostgreSQL, SVN}
    \\\\\textbf{2014 - Stage de 3 mois} \\
    \textsc{CCM Benchmark} -
     \textit{PME à Rennes (35)}
    \\\\Analyse, développement et mise en production de l'application \textit{VideoSnack} qui permet de visualiser des vidéos sur palteforme \textit{Android}.
    \\\\\textit{SDK Android, Eclipse, Java, JSON, API REST, PHP, Bootstrap, Agile, Jira, Git, PostgreSQL}
    \\\\\textbf{2013 - Stage de 3 mois} \\
    \textsc{Acro-Caval} -
    \textit{Association à Nîmes (30)}
    \\\\Analyse, développement et mise en production du site Internet de l'association.
    \\\\\textit{UML, Codeigniter, PHP , HTML/CSS, JQuery, MySQL, Bootstrap, OVH}
    \\\\\framebox[8.7cm]{\textbf{Intérêts}}\\\\
    Développement informatique, arts plastiques,
    cirque et activités sportives.
    \\\\\framebox[8.7cm]{\textbf{Informations}}\\
    \begin{description}%[align=margin,labelsep=0pt]
      \item[Date de naissance] 17/10/1992
      \item[Téléphone] 07 83 76 35 76
      \item[Email] thibault.gauthier.dev@gmail.com
      \item[Site internet] thibault.ovh
    \end{description}

  \end{minipage}





  %en dehors de l'env minipage
  \addtocounter{footnote}{-2} %3=n
  \stepcounter{footnote}\footnotetext{Méthodes Informatiques Appliquées à la Gestion des Entreprises}
  \stepcounter{footnote}\footnotetext{Systèmes d'Informations d'Aide à la Décision}
  \stepcounter{footnote}\footnotetext{Sciences de l'Ingénieur}
\end{titlepage}
