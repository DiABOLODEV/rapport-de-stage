\documentclass[a4paper, 12pt, onecolumn, openright, oneside]{report}

\usepackage[utf8]{inputenc}
\usepackage[pdftex]{graphicx}
\usepackage{setspace}
\usepackage[T1]{fontenc}
\usepackage[francais]{babel}
\usepackage{color}
\usepackage{enumitem}
\usepackage{graphicx}
\usepackage{fancybox}
\usepackage{color}
\usepackage{geometry}
\usepackage{comment}
\usepackage{lipsum}
\usepackage[french]{minitoc}
\usepackage[nottoc]{tocbibind}
\usepackage[nottoc]{tocbibind}
\usepackage{fancyhdr} %pour definir les entêtes et pied de pages
\usepackage[Sonny]{fncychap} %style de chapitre
\usepackage{caption} %caption des images figure
\usepackage{epigraph} %citations%
\usepackage{lmodern}
\usepackage{dirtree}
% The following is a dummy icon command
\newcommand\myicon[1]{{\color{#1}\rule{2ex}{2ex}}}
% If you have actual icon images, use \includegraphics to include them
% If you are generating them, put in the appropriate code for them here
% now we make a command for a folder/file which inserts the icon and its label
% adjust this as needed. If you only have 2 icons, then you could create
% a \myfile and \myfolder command with the icon fixed.
\newcommand{\myfolder}[2]{\myicon{#1}\ {#2}}

\usepackage{pict2e} %shéma%
\pagestyle{fancy}
\fancyhead[C]{}
\fancyhead[L]{\leftmark}
\fancyhead[R]{}
\definecolor{rouge}{RGB}{255,112,119}
\definecolor{rouge-clair}{RGB}{255,163,168}
\definecolor{rouge-tres-clair}{RGB}{255,217,219}
\newcommand{\hr}{\rule{\linewidth}{0.5mm}}
\newcommand{\br}{\\[0.5cm]}
\renewcommand*{\familydefault}{\sfdefault}
%Numéro de sections dans la marge
\makeatletter
\def\@seccntformat #1{ %
\protect\makebox[0pt][r]{\csname the#1\endcsname \quad }}
\makeatother
\captionsetup{
font=footnotesize,
justification=raggedright,
singlelinecheck=false
}
\usepackage{xcolor}
\usepackage{adjustbox}

\newenvironment{colbox}[2]{%
    \begin{adjustbox}{minipage=[b]{380px},margin=1ex,bgcolor=#1,env=center}% or use `bgcolor={HTML}{#1}` if you want to force HTML colors
        \textbf{#2}\\
}{%
    \end{adjustbox}%
}

\begin{document}
  \begin{titlepage}
  \begin{minipage}[t]{5cm} %t = top%
    \flushleft \includegraphics[width = 7cm]{images/logo_univrennes1.png}
    \hfill \noindent
  \end{minipage}
  %
  %\begin{minipage}[t]{7cm}
  %  \flushright \includegraphics[width = 5cm]{images/logo_capgemini.png}
%  \end{minipage}
  \vspace*{\fill}
  \begin{center}
    \textsc{Thibault Gauthier},\\ étudiant en \textsc{Master 2 MIAGE\footnote{Méthodes informatiques appliquées à la gestion des entreprises}} à l'\textsc{ISTIC\footnote{L'ISTIC est une unité de formation et de recherche
    en INFORMATIQUE et ÉLECTRONIQUE de l'université de Rennes 1 -http://www.istic.univ-rennes1.fr/}}\\a l'honneur de vous présenter son\\[0.4cm]
    \hr\\[0.5cm]
    {\huge\textbf{Rapport de stage }}\\[0.4cm]
    Au sein de la société \textsc{Capgemini} de Rennes sur le sujet du\\[0.4cm]
    {\large\textbf{Développement informatique sur le projet Geofibre}}\\[0.4cm]
    \hr\\[0.5cm]
    Mise à jour le \today
  \end{center}
  \vspace*{\fill}
\end{titlepage}

  %-------------------------------------------------------%
  \begin{flushright}
Je tiens à remercier toutes les personnes qui ont contribué au bon déroulement de mon stage.
\end{flushright}

  %-------------------------------------------------------%
  \setcounter{secnumdepth}{1} %pas de numérotation des subsections
  \setcounter{tocdepth}{1}  %pas de numérotation des subsections
  \tableofcontents
  %-------------------------------------------------------%
  \chapter*{Introduction}
\addstarredchapter{Introduction}
La fin du master MIAGE se concrétise par la réalisation d’un stage en entreprise d’une durée de 6 mois.
\\J’ai choisi de réaliser ce stage au sein de l’entreprise \textsc{Capgemini} du 9 mars au 21 août 2015,
dans le centre de services TMA OSS\footnote{Tierce Maintenant Applicative des applications orientés réseau d'Orange} à Rennes.
\\Mon choix s’est porté sur ce stage pour plusieurs raisons.
 Tout d’abord, il s’agit d’intégrer une équipe de travail ; mes différentes expériences de stage ayant jusqu’alors été réalisées en
autonomie ou en binôme.
De plus, j’ai pu évaluer concrètement la gestion de projet sur un
projet d’envergure.
  Ensuite, cela m’a permis de monter en compétence dans le domaine SIG\footnote{Systéme d'Information Géographique} et
sur les outils et technologies liés au projet sur lequel j'ai travaillé.
\\Pour finir, ces six mois de stage m’ont permis d’apprécier le
fonctionnement d’une ESN\footnote{Entreprise de services du numérique}
, secteur dans lequel je n’avais pas encore réalisé de stage.
\\\\
Dans un premier temps je présenterai le contexte du stage, l'entreprise d'accueil et le projet sur lequel j'ai travaillé.
Puis, dans un second temps je présenterai le stage en lui même.

  %-------------------------------------------------------%
  \part{Contexte du stage}
  \chapter{La soci�t� Capgemini}
  \section{Historique}
  \section{Positionnement par rapport � la concurrence}
  \section{Le site de Rennes}
  \subsection{Le p�le TMA OSS}
  \chapter{La soci�t� Capgemini � Rennes}
\section{Histoire}
\section{Solutions et services}
\section{L'unit� TMA OSS}
  \chapter{Présentation}

	\section{Objet}
Le projet Geofibre a pour objet de fournir une application de \textsc{Système d'Information Géographique (SIG)} pour le FTTH\footnote{Fiber To The Home. C'est le réseau trés haut débit de fibre optique pour les clients résidentiels.}\\
Geofibre propose une application WEB permettant la gestion des données descriptives du réseau de fibre optique en France pour les clients résidentiels.\\
L'application doit supporter le déploiement du réseau FTTH en termes de conception et de gestion du réseau. \\
\\
L'application repose sur le moteur SIG ArcGIS 10.0 de la société ESRI et propose une application Web accessible à travers le GASSI de Orange.
\begin{figure}[h]
  \captionbox{Logo de l'application Geofibre\label{fig:dummy}}{
    \includegraphics[width=2cm]{images/logo_geofibre.png}
  }
\end{figure}
	\section{Historique}
	Le projet est née en 2010. Les deux premières années le projet s'est développé en méthode AGILE dans les locaux du client Orange à Lannion.
	Les employés de Capgemini étaient présents pour l'assistance technique.


\begin{colbox}{{HTML}{A3E8FF}}{Le versionnage des applications Capgemini: }
    What is ....
\end{colbox}


	\chapter{Acteurs}
		\section{Client \textsc{Orange}}
		\section{Le prestataire \textsc{Capgemini}}
	\chapter{Méthodologie de travail}
	\section{Identification des versions}

	Les versions sont marquées par des labels qui doivent permettre d'identifier de façon non équivoque toutes les évolutions successives des composants pour pouvoir retrouver et extraire de la base d'archives toute version livrée au client ou livrée pendant les phases d'intégration ou de la validation interne.
	\\\\
	On distingue deux types de versions :
	\begin{description}
		\item[Version majeure] : c'est une version complète du logiciel, c'est à dire qu'elle contient l'ensemble des composants du système
		\item[Version mineure] : c'est une version paertielle du logiciel, c'est à dire qu'elle ne contient qu'un sous-ensebmel des composants du système, qui constitue un delta par rapport à la version précédente
		(qui peut être une version majeure ou mineure) ; c'est en général le résultat d'une correction ou d'une évolution mineure.
	\end{description}
	Les labels de version sont structurés de telle sorte que cette dépendance entre versions soit mise en évidence.
	\\La composition d'un label de version est de la forme \textsc{GxxRyyCzz}.
	\\Dans ce sigle on retrouve :
	\begin{description}
		\item[Révision] : Une révision est attachée à un composant. \'A chaque fois qu'un utilisateur archive une nouvelle version d'un composant, l'outil de gestion de configuration crée une nouvelle révision de ce composant.
		\item[Version et labels] : Une version permet d'identifier un ensemble cohérent de composants d'une application. L'identifiant de version est sous contrôle complet de l'équipe de projet. Par exemple la première version est la G1R0C0, puis les suivantes seront les
		G1R1C0 puis la G2R0C0.
		\item[Tronc et branches] : Le \textit{tronc} supporte les versions principales. En cas de travaux parallèles sur plusieurs versions (par exemple la correction d'une anomalie sur une version n-1 et développement de la version n), on crée une branche qui va permettre de modifier une version déjà livrée.
		\\
		 \end{description}
\textit{Exemple} : La branche G1R0 contient les versions correctives G1R0C1 et G1R0C2 qui intégrent des correctifs d'anomalies idnetifiées sur la version G1R0C0 préalablement livrée.
\setlength{\unitlength}{1.3cm}

\begin{picture}(5,5)
		%traits haut

   \put(2.5,4.4){\line(1,0){1.5}}
	 \put(6.5,4.4){\line(1,0){1.5}}
	 \put(10.5,4.4){\line(1,0){1.5}}
	 %oblique
	 \put(1.5,4){\line(0.3,-1){0.5}}
	 %traits bas
	 \put(4.5,2.4){\line(1,0){1.5}}
	 %fleches
	 \put(4,4.4){\line(-0.3,0.3){0.2}}
	 \put(4,4.4){\line(-0.3,-0.3){0.2}}

	 \put(8,4.4){\line(-0.3,0.3){0.2}}
	\put(8,4.4){\line(-0.3,-0.3){0.2}}

	\put(12,4.4){\line(-0.3,0.3){0.2}}
	\put(12,4.4){\line(-0.3,-0.3){0.2}}

	\put(6,2.4){\line(-0.3,0.3){0.2}}
	\put(6,2.4){\line(-0.3,-0.3){0.2}}

	%fleche oblique
	\put(2,2.4){\line(-0.3,0.2){0.2}}

   \put(0,4){\framebox(2.5,0.8)[c]{G1R0C0}}
	 \put(4,4){\framebox(2.5,0.8)[c]{G1R1C0}}
	 \put(8,4){\framebox(2.5,0.8)[c]{G2R0C0}}

	 \put(2,2){\framebox(2.5,0.8)[c]{G1R0C1}}
	 \put(6,2){\framebox(2.5,0.8)[c]{G1R0C2}}

\end{picture}

  %-------------------------------------------------------%
  \part{Réalisation du stage}
  \chapter{Le Stage}
\section{Sujet}

Le sujet de stage est la participation au développement d'une évolution sur l'application Geofibre.
Le client souhaiterais en effet intégrer de nouvelles fonctionnalités, notamment l'intégration des cartes des départements d'Outre-Mer au sein de l'application.
\\Cette version s'annoncant conséquente, l'équipe dirigante a décidé de renforcer le groupe.

\section{Objectif}


L'objectif du stage est, dans un premier temps, de participer aux phases de développement jusqu'à la livraison pour l'évolution prévue sur l'application et dans un second temps de participer à la maintenance de l'application.

\chapter{Organisation de l'équipe}

L'équipe de travail est organisée de la façon suivante :\\

\begin{itemize}
  \item Le chef de groupe
  \item Le chef de projet
  \item L'équipe de développement
  \item Le responsable de test, c'est aussi le chef de projet.
  \item Le responsable du groupe, c'est un membre de l'équipe de développement.
  \item Le responsable du groupe 2, c'est un membre de l'équipe de développement.\\
\end{itemize}
Je suis intégré au sein de l'équipe de développement composée de 10 personnes (1 externe et 9 salariés).
\\Nous fonctionnons suivant la méthode \textit{LEAN}.

 \begin{colbox}{{HTML}{A3E8FF}}{La méthode LEAN\\ }
   \textbf{Objectif} : Améliorer de façon continue la performance en termes de qualité, coûts et délais de livraison.
   \\\textbf{Origine} : Apparue dans le seconde moitié du XXème siècle avec l'entreprise \textit{Toyota}. La production de voiture répond à une demande, ainsi les stocks sont quasi inexistants.
   \\\textbf{Principe} : Créer de la valeur ajoutée pour le client avec un minimum de gaspillage et en livrant un maximum de qualité.
   \\\textbf{En tant que développeur} : Des indicateurs de qualité de code à améliorer au fil des versions, des délais de livraison du service fini à respecter.
 \end{colbox}

 Chaque jour à 9h30 nous avons une réunion (\textit{Daily Meeting}) où chaque membres de l'équipe, tour par tour, décrit son humeur de la veille, les tâches qu'il a réalisé, les problèmes éventuels à signaler et ce qu'il prévoit de faire au fil de la journée.
 \\ C'est une méthode qui permet de savoir où en est le projet, et plus particuliéremment chaque membres de l'équipe. \\Cette méthode permet aussi de chercher les solutions ensemble aux problèmes et affecter plus de personnes sur une tâche bloquante dans la limite du possible.
\\La communication et la transparence sur le travail réalisé font que les problèmes ne restent pas longtemps sans solutions.
\\De plus, l'aménagement de l'\textit{Open-Space} permet de demander de l'aide rapidement aux collègues de travail. Ca permet de ne pas rester bloquer sur une tâche ou se désorienter.
\\Le chef de projet écrit des fichiers de suivi d'avancement des tâches pour chaque phases du cycle du projet. Aux développeurs de le remplir en indiquant le temps passé sur chaques tâches réalisées et d'évaluer le \textit{RAF\footnote{le Reste à Faire}}. De cette manière le chef de projet et le chef de groupe peuvent planifier et piloter avec des risques moindre la suite du projet.

\chapter{Environnement technique}
\section{Architecture technique}
L'architecture technique repose sur des machines virtuelles (excepté la textit{BDD}).
Voici la liste des infrastructures présentes :
\begin{description}
\item[Serveur WAS] C'est le serveur qui délivre l'application à l'utilisateur. En effet, l'utilisateur s'y connecte via le \textit{GASSI}\footnote{Gestionnaire d'Accès Sécurisé interne au Système d'Information} du client avec le protocole \textit{HTTPS}\footnote{Hypertext Transfer Protocol Securised} ou via un \textit{VPN}\footnote{Virtual Private Network} avec le protocole \textit{SSL}\footnote{Secure Sockets Layer}.
\\Il fonctionne sur une machine Linux avec le serveur d'application \textit{JOnAS}\footnote{Java Open Application Server}.
\item[Serveur ArcGIS] Basé sur le progiciel \textit{ArcGIS} de l'éditeur \textit{ESRI}, il permet de traiter les données \textit{SIG} (calcul de projection, géométries ...). Il délivre les informations récoltées et traitées à partir de la base de données au serveur \textit{WAS} sur la base d'une architecture \textit{REST}\footnote{REpresentional State Transfer}; Il communique aussi avec des interfaces externes, par exemple avec l'application \textit{Sigeo} (développé par \textsc{Capgemini}) pour récupérer les \textit{tuiles}\footnote{Images de fond de plan et images du cadastre}.
\item[Serveur d'impression] En raison de la charge induite par la génération des documents (\textit{PDF}) destiné à l'impression de fond de plan (certains au format A0), des serveurs sont dédiés à cette tâche. Il fonctionne eux aussi avec le progiciel \textit{ArcGIS}.
\item[Serveur SGBD] Le serveur de base de données est \textit{PostGreSQL} et permet de gérer l'accès et le stockage des données.\\
\end{description}

\'Etant donné la charge sur l'application (rappel : 1150 utilisateur simultanés par jour) il existe plusieurs instances de serveurs et la communication d'un serveur à un autre se fait via des répartiteurs de charges qui vont requêter le bon serveur au bon moment afin d'équilibrer la charge de travail entre les différents serveurs. De ce fait il y a, en plateforme de production :\\

\begin{itemize}
	\item 3 Serveurs WAS
	\item 8 Serveurs ArcGIS pour la France Métropolitaine et 2 pour les DOM
	\item 1 Serveur de base de données
	\item 4 Serveurs d'impressions\\
\end{itemize}
\section{Outils et technologies}
\subsection{Adobe FlashBuilder}
C'est un envirronement de développement d'applications basé sur le langage \textit{Actionscript} et le \textit{framework Flex Open Source}.
 \\On l'utilise pour développer et débuguer l'application \textit{frontoffice} qui sera plaçée sur le serveur WAS.
\subsection{Mozilla Firefox}
C'est un navigateur web.
 Il permet d'accèder à l'application via l'URL du serveur qui délivre une page HTML avec l'application \textit{frontoffice} embarquée dans un objet \textit{Flash}.
 \\Aussi on utilise le plugin \textit{Firebug} qui permet de voir les requêtes HTTP envoyées et reçues par l'application,
 ça permet de débuguer les communications avec le serveur.
\subsection{Eclipse} Eclipse est un environnement de développement basé sur la langage \textit{Java}. Nous utilisons un environnement JEE afin de développer et débuguer le \textit{backoffice} qui intégre le SDK ArcGIS et qui permet de faire
 les tâches relatives au SIG.
\subsection{Qgis Desktop} C'est un logiciel qui permet de visualiser des données SIG. On l'utilise pour vérifier si des données sont biens représentées dans les phases de tests ou pour construire des jeux de données.
\subsection{PgAdmin} C'est une interface d'administration à la base de données PostgreSQL utilisé par le projet.
\subsection{shell Linux} Afin d'accèder aux serveurs WAS, ArcGIS ou SGBD via ssh et lancer différents scripts sur les machines (par exemple il y a un script pour la copie de données d'une commune à une autre).

\chapter{Configuration du projet}
\section{Identification des versions}
\label{versionning}
Les versions sont marquées par des labels qui doivent permettre d'identifier de façon non équivoque toutes les évolutions successives des composants pour pouvoir retrouver et extraire de la base d'archives toute version livrée au client ou livrée pendant les phases d'intégration ou de la validation interne.
\\\\
On distingue deux types de versions :
\begin{description}
	\item[Version majeure] : c'est une version complète du logiciel, c'est à dire qu'elle contient l'ensemble des composants du système
	\item[Version mineure] : c'est une version paertielle du logiciel, c'est à dire qu'elle ne contient qu'un sous-ensebmel des composants du système, qui constitue un delta par rapport à la version précédente
	(qui peut être une version majeure ou mineure) ; c'est en général le résultat d'une correction ou d'une évolution mineure.
\end{description}
Les labels de version sont structurés de telle sorte que cette dépendance entre versions soit mise en évidence.
\\La composition d'un label de version est de la forme \textsc{GxxRyyCzz}.
\\Dans ce sigle on retrouve :
\begin{description}
	\item[Révision] : Une révision est attachée à un composant. \'A chaque fois qu'un utilisateur archive une nouvelle version d'un composant, l'outil de gestion de configuration crée une nouvelle révision de ce composant.
	\item[Version et labels] : Une version permet d'identifier un ensemble cohérent de composants d'une application. L'identifiant de version est sous contrôle complet de l'équipe de projet. Par exemple la première version est la G1R0C0, puis les suivantes seront les
	G1R1C0 puis la G2R0C0.
	\item[Tronc et branches] : Le \textit{tronc} supporte les versions principales. En cas de travaux parallèles sur plusieurs versions (par exemple la correction d'une anomalie sur une version n-1 et développement de la version n), on crée une branche qui va permettre de modifier une version déjà livrée.
	\\
\end{description}
\textbf{Exemple} : La branche G1R0 contient les versions correctives G1R0C1 et G1R0C2 qui intégrent des correctifs d'anomalies idnetifiées sur la version G1R0C0 préalablement livrée.
\setlength{\unitlength}{1.3cm}

\begin{picture}(5,5)
	%texte
	\put(-2,4.4){Tronc}
	\put(-2,2.4){Branche G1R0}
	%traits haut
	\put(2.5,4.4){\vector(1,0){1.5}}
	\put(6.5,4.4){\vector(1,0){1.5}}
	\put(10.5,4.4){\vector(1,0){1.5}}
	%oblique
	\put(1.5,4){\vector(0.3,-1){0.5}}
	%traits bas
	\put(4.5,2.4){\vector(1,0){1.5}}
	\put(0,4){\framebox(2.5,0.8)[c]{G1R0C0}}
	\put(4,4){\framebox(2.5,0.8)[c]{G1R1C0}}
	\put(8,4){\framebox(2.5,0.8)[c]{G2R0C0}}
	\put(2,2){\framebox(2.5,0.8)[c]{G1R0C1}}
	\put(6,2){\framebox(2.5,0.8)[c]{G1R0C2}}
\end{picture}
\begin{colbox}{{HTML}{C7FF99}}{}
Durant mon stage j'ai participé à l'intégration de la 6ème version (G1R6C0) et au développement et à l'intégration de la 7ème version (G1R7C0).
\end{colbox}

\newpage

\section{Organisation des environnements de travail}

Le \textbf{référenciel} (\textit{Repository}) contient l'ensemble des révisions de chaque composant ainsi que les liens entre composants permettant d'identifier les versions successives de chaque application.
\\\\
Les \textbf{espaces de travail} (\textit{Workspaces}) sont les espaces utilisés pour développer, intégrer, valider et livrer chaque application.
\\
\begin{picture}(0,1)
	%Referenciel
	\put(0,-2){Référenciel}
	\put(0.7,-1.5){\oval(2,2)[t]}
	\put(-0.3,-2.5){\line(0,1){1}}
	\put(1.7,-2.5){\line(0,1){1}}
	\put(0.7,-2.5){\oval(2,2)[b]}
	%fleches
	\put(1.7,-1.5){\vector(1,0){6}}
	\put(2.6,-1.4){Extraction des composants}

	\put(7.7,-2.5){\vector(-1,0){6}}
	\put(2.6,-2.4){Archivage des composants}
	%espaces de travail x+8
	\put(7.8,-2){Espaces de travail}
	\put(8.2,-2.6){Composants}
	\put(8.7,-1.5){\oval(2,2)[t]}
	\put(7.7,-2.5){\line(0,1){1}}
	\put(9.7,-2.5){\line(0,1){1}}
	\put(8.7,-2.5){\oval(2,2)[b]}
	%+0.3
	\put(9,-1.5){\oval(2,2)[t]}
	\put(8,-2.5){\line(0,1){1}}
	\put(10,-2.5){\line(0,1){1}}
	\put(9,-2.5){\oval(2,2)[b]}

	\put(9.3,-1.5){\oval(2,2)[t]}
	\put(8.3,-2.5){\line(0,1){1}}
	\put(10.3,-2.5){\line(0,1){1}}
	\put(9.3,-2.5){\oval(2,2)[b]}

\end{picture}
\\[6cm]
Quand l'activité le justifie, il est possible de devoir travailler simultanément sur plusieurs versions, en général :
\begin{itemize}
	\item Une version en \textbf{développement}
	\item Une version en \textbf{maintenance}\\
\end{itemize}
Il faut donc prévoir autant d'espaces de travail disponibles et ceci pour les différentes phases du cycle de développement :
\begin{itemize}
	\item Développement et tests unitaires
	\item Intégration et validation
	\item Livraison (effectuée sur la plate-forme de qualification)\\
\end{itemize}

Geofibre est versionné avec SVN\footnote{Subversion}, voici la hiérarchie des projets du \textit{repository} :\\

\dirtree{%
.1 \myfolder{red}{Trunk }.
.2 \myfolder{black}{gfi-front \textit{frontoffice}}.
.3 \myfolder{black}{5 projets}.
.2 \myfolder{black}{gfi-back \textit{backoffice}}.
.3 \myfolder{black}{7 projets}.
.2 \myfolder{black}{gfi-bdd \textit{base de données}}.
.3 \myfolder{black}{3 projets}.
.2 \myfolder{black}{gfi-expl textit{exploitation}}.
.3 \myfolder{black}{9 projets}.
}


\chapter{\'Evolutions et mise en place}
\section{Version G1R6}

La version applicative G1R6 de Geofibre doit permettre la prise en compte des DOMs. Pour cela des instances spécifiques sont mises en place  pour les différents départements (Réunion, Martinique, Guadeloupe et Guyane).
La mise à disposition de Geofibre dans les DOMs doit être équivalente vue de l’utilisateur à la version métropole.
\\Les données dans les DOMs seront gérées dans le système de projection local. Il n’y aura pas, comme en métropole (Lambert II étendu vers Lambert 93), de reprojection vers le système local ou d’export de données vers un autre système.\\

\begin{tabular}{|l|c|r|}
  \hline
    Zone & Système géodesique & Projection \\
  \hline
  France métropolitaine & RGF93 & Lambert 93 \\
  Guadeloupe & WGS84 & UTM Nord fuseau 20 \\
  Martinique & WGS84 & UTM Nord fuseau 20 \\
  Guyane & RGFG95 & UTM Nord fuseau 22 \\
  Réunion & RGR92 & UTM Sud fuseau 40 \\
  \hline
\end{tabular}\\\\

Malgré le fait que les serveurs soient hébergés en métropole, les horaires de création ou modification des objets stockés en base DOMs seront renseignés en heure locale.

\section{Version G1R7}
Cette version est essentiellement  fonctionnelle et dédiée à la prise en comptes de des paliers RIP\footnote{Les réseaux d’initiative publique} et DSP\footnote{Délégation de Service Public}.

\section{Cycle de développement en V}
Le projet fonctionne en cycle en V, suivant ce schéma :\\
\noindent%
\begin{minipage}{\linewidth}% to keep image and caption on one page
\makebox[\linewidth]{%        to center the image
  \includegraphics[keepaspectratio=true,scale=0.8]{images/cycle_en_v.png}}
\captionof{figure}{Cycle de développement en V}\label{visina8}%      only if needed
\end{minipage}

\section{Plannification}
Voici le planning qui représente la répartition des tâches durant mon stage :\\\\
\noindent%
\begin{minipage}{\linewidth}% to keep image and caption on one page
\makebox[\linewidth]{%        to center the image
  \includegraphics[keepaspectratio=true,scale=0.8]{images/gant.png}}
\captionof{figure}{Planning des tâches réalisées}\label{visina8}%      only if needed
\end{minipage}
\chapter{Travail réalisé}
Je suis arrivé sur le projet lorsque la version G1R6 étais à la fin de la phase de programmation. J'ai participé un peu à la programmation, à la phase d'intégration et de maintenance.
\\J'ai pu commencer la version G1R7 de la rédaction des spécifications jusqu'à l'intégration.
\section{Développement}
\subsection{Version G1R6}
On m'a rapidement permis de développer sur le projet. La première tâche consisté à externaliser des paramètres de configuration concernant le zoom, la projection et la minicarte qui se trouve "en dur" dans l'application.
\\Cette première tâche de développement m'a permis de mieux comprendre le fonctionnement du projet car j'ai du modifier les trois parties de l'application :\\
\begin{itemize}
\item La base de données (gfi-bdd). En insérant de nouvelles données dans la table de configuration
\item Le backoffice (gfi-back). En faisant le Mapping\footnote{Association des données en base à des objets en programmation} des données.
\item Le frontoffice (gfi-front). En supprimant les données de configuration "en dur" dans le programme et en envoyant les bonnes commandes au serveur pour récupérer les paramètres de configurations présents en base de données.\\
\end{itemize}
J'ai posé beaucoup de questions à l'équipe de développement pour valider mon travail.
\\ Ensuite j'ai eu en charge de vérifier si le paramètre de projection était bien transmis aux  \textit{Toolboxs\footnote{Les Toolboxs sont des servlets Java, ce sont des extensions des fonctions du serveur de base.}} (gfi-back) et si elles étaient bien aiguillés en fonction de ce paramètre de projection.
Pour cela j'ai du vérifier les requêtes envoyées par le client lors de l'appel de la Toolbox et vérifier dans le backoffice si le paramète était bien transmis et vers la bonne servlet en analysant ce qui était retourné.

\subsection{Version G1R7}
Lors de la phase de développement de la version G1R7 j'avais beaucoup plus d'expérience sur le projet et je n'ai pas eu de  gros problèmes à développer ce qui été demander et j'avais déjà les idées sur l'implémentation de ce que je devais faire.
\\En effet je me suis occupé du widget \textit{Publication de Schéma Directeur} (gfi-front, gfi-back, gfi-data) où j'ai ajouté le champs opérateur au niveau de la base de données et de l'IHM. Puis j'ai modifier les commandes de selections, d'extraction et d'impression pour qu'elles se réalise avec un filtrage sur le champ opérateur.
\\\\
\noindent%
\begin{minipage}{\linewidth}% to keep image and caption on one page
\makebox[\linewidth]{%        to center the image
  \includegraphics[keepaspectratio=true,scale=0.5]{images/publicationSD.png}}
\captionof{figure}{Widget publication SD}\label{visina8}%      only if needed
\end{minipage}\\

Je me suis ensuite occupé du \textit{programme de copie de données} (gfi-expl), qui permet de copier des données FTTH d'une commune à une autre.
\\J'ai ajouté une contrainte lié à la configuration des RIP : si la commune d'export n'a pas de configuration RIP alors elle prend pour valeur la configuration RIP de la table source.

\section{Intégration}
Je parlerais des tests d'integration, leurs niveau d'importance, le système de vagues, et un exemple de passage
\section{Corrections d'anomalies}
Je parlerais de comment on corrige une anomalie et un exemple.

  %-------------------------------------------------------%
  \chapter*{Résumé}
\addstarredchapter{Résumé}
Dans le cadre du Master 2 MIAGE j'ai choisi de réaliser un stage de 6 mois dans la société \textsc{Capgemini} à Rennes.
\\\\
Au sein du service TMA OSS j'ai participé à la maintenance et au développement d'évolutions sur l'application Geofibre.
Cette application de SIG permet aux chargés d'affaires, via une IHM web de gérer et de concevoir le réseau FTTH domestique en France métropolitaine.
\\\\
Mon rôle a été d'apporter du soutient à l'équipe de développement pour l'intégration d'une nouvelle version de l'application
permettant de gérer le réseau FTTH des départements d'Outre-Mer (Guyane, Guadeloupe, Martinique, Réunion).
Ensuite, durant la phase de spécifications de la future version je me suis consacré à corriger des anomalies relevées par le client.
Enfin j'ai participé au développement et à l'intégration de la dernière version de l'application qui apporte la gestion de nouvelles données d'opérateurs.

  \chapter*{Resume}
\addstarredchapter{Resume}

In english please

  %-------------------------------------------------------%
  \chapter*{Conclusion}
\addstarredchapter{Conclusion}

Pour commencer le projet Geofibre est très intéressant de bout en bout et pour commencer m'a permis de découvrir le monde du SIG et du FTTH.
\\Techniquement j'ai pu découvrir le langage ActionScript et le framework Flex avec la programmation évenementiel même si c'est une technologie de moins en moins utilisée.
 \\ J'ai aussi pu découvrir le fonctionnement et l'utilité du progiciel ArcGIS.
 \\Aussi j'ai appris beaucoup de petites techniques de développement et de débuguage, notamment sur l'IDE Eclipse/FlashBuilder et en PostgreSQL.
\\\\Professionnellement cela m'a permis d'enrichir mon expérience et de mieux comprendre les différentes phases de déroulement d'un projet, ainsi que le développement et la méthodologie de travail dans un contexte industriel.
\\\\En ce qui concerne mon apport pour l'entreprise \textsc{Capgemini}, les évolutions développés et testés pour la version G1R6 ont été livrées à Orange et vont être utilisé en production.
\\\\Le stage m'a beaucoup apporté. J'ai travaillé avec une jeune équipe, qualifiée et compétente qui m'ont énormément aidé. J'ai pu apprendre, observer et acquérir de nouvelles compétences mais aussi apporter le fruit de mes années d'études au projet.

  %-------------------------------------------------------%
  \appendix
  \listoffigures
  \chapter{Bibliographie / Webographie}
\begin{description}
\item{[1]} https://www.cadastre.gouv.fr
\item{[2]} http://ostermiller.org/
\item{[3]} [..]
\end{description}

  \chapter{Carnet de bord des travaux réalisés par semaine}
\label{travauxreal}
\begin{enumerate}[label= Semaine \no\textbf{\arabic*.},itemsep=20pt]
\setcounter{enumi}{10}

\item \textbf{\colorbox{rouge}{Début du stage. }}\\\\
\textbf{Visite, présentation et rencontre} avec les équipes de la ferme d'applications \textsc{TMA OSS\footnote{Tierce Maintenant Applicative des applications orientés réseau d'Orange}}. Explication de l'activité par le chef de service \textsc{Arnaud Bellina}.
\newline Visite, présentation et rencontre avec les différents services du bétiment de Capgemini (Infirmerie, CE, Cafétaria, RH, Assistante) .
\newline \textbf{Installation de mon poste de travail} au sein de l'openspace de l'équipe Géofibre et intégration supervisée par le chef de groupe \textsc{Patrick Veillon} et la chef de projet \textsc{Anne-Sophie Lescop}.
\newline \textbf{Installation des logiciels} et \textbf{lecture} de la documentation ainsi que du code qui compose le projet Géofibre épaulé par l'équipe.

\item \textbf{Montée en compétence} générale sur l'application Géofibre.
\item \textbf{Développement de la version G1R6 Front (IHM Flex)} Externalisation des systémes de projection, emprise, échelles, minimap

\item \textbf{Développement de la version G1R6 Back (Serveur, Toolbox)} Vérification de la gestion de la projection

\item \textbf{Développement de la version G1R6 Back (Serveur, Toolbox)} Aiguillage servlet
\newline \textbf{Développement de la version G1R6 Back (Serveur, Toolbox)} Impact code appelant

\item \textbf{Tests d'intégration de la version G1R6 sur la Réunion}
\begin{enumerate}[label = Tests \no\arabic*.,align=left]
\item \emph{\colorbox{rouge}{P1}} - Gestion infrastructure - Recalage GC
\item \emph{\colorbox{rouge}{P1}} - Gestion infrastructure - Zone de recalage
\item \emph{\colorbox{rouge-clair}{P2}} - Exploitation - Import RCV (Référenciel Commune Voies)
\item \emph{\colorbox{rouge-clair}{P2}} - Localisation adresse
\item \emph{\colorbox{rouge-tres-clair}{P3}} - Purge des fichiers (multi instance)
\end{enumerate}

\item \textbf{Tests d'intégration de la version G1R6 sur la Guyane}
\begin{enumerate}[label = Tests \no\arabic*.,align=left]
\item  \emph{\colorbox{rouge}{P1}} - Gestion FTTH - Cébles
\item \emph{\colorbox{rouge}{P1}} - Gestion FTTH - Parcours
\item\emph{\colorbox{rouge}{P1}} -  Gestion FTTH - Zone de travail
\item \emph{\colorbox{rouge}{P1}} - Gestion infrastructure - Itinéraires GC
\item \emph{\colorbox{rouge}{P1}} - Gestion infrastructure - Site supports
\item \emph{\colorbox{rouge-tres-clair}{P3}} - Filtrage
\item \emph{\colorbox{rouge-tres-clair}{P3}} - Gestion des droits
\item \emph{\colorbox{rouge-tres-clair}{P3}} - Géosignets
\item \emph{\colorbox{rouge-tres-clair}{P3}} - Outil de mesure
\item \emph{\colorbox{rouge-tres-clair}{P3}} - Sauvegarde du contexte
\item \emph{\colorbox{rouge-tres-clair}{P3}} - Table des matiéres
\end{enumerate}

\item \textbf{Tests d'intégration de la version G1R6 sur la Guadeloupe}
\begin{enumerate}[label = Tests \no\arabic*.,align=left]
\item \emph{\colorbox{rouge}{P1}} - Gestion infrastructure - Site supports
\item \emph{\colorbox{rouge}{P1}} - Exports - Dossier OPGC - Base arriére de PM
\item \emph{\colorbox{rouge-clair}{P2}} - Méj adresse des immeubles depuis optimum
\item \emph{\colorbox{rouge-clair}{P2}} - Exploitation - majBatchData
\item \emph{\colorbox{rouge-clair}{P2}} - Statistiques
\item \emph{\colorbox{rouge-tres-clair}{P3}} - Filtrage
\item \emph{\colorbox{rouge-tres-clair}{P3}} - Gestion des droits
\item \emph{\colorbox{rouge-tres-clair}{P3}} - Géosignets
\item \emph{\colorbox{rouge-tres-clair}{P3}} - Localisation objet métier
\end{enumerate}

\item \textbf{Tests d'intégration de la version G1R6 sur la Martinique}
\begin{enumerate}[label = Tests \no\arabic*.,align=left]
\item  \emph{\colorbox{rouge}{P1}} - Gestion FTTH - Cébles
\item \emph{\colorbox{rouge}{P1}} - Gestion FTTH - Parcours
\item \emph{\colorbox{rouge}{P1}} - Gestion infrastructure - Itinéraires GC
\item \emph{\colorbox{rouge}{P1}} - Gestion FTTH - Projets
\item \emph{\colorbox{rouge}{P1}} - Gestion FTTH - Schéma directeur
\item \emph{\colorbox{rouge}{P1}} - Gestion FTTH - Régles d'ingienerie
\item \emph{\colorbox{rouge}{P1}} - Décalages horaires
\item \emph{\colorbox{rouge-clair}{P2}} - Statistiques
\item \emph{\colorbox{rouge-tres-clair}{P3}} - Filtrage
\item \emph{\colorbox{rouge-tres-clair}{P3}} - Gestion des droits
\item \emph{\colorbox{rouge-tres-clair}{P3}} - Outil de mesure
\item \emph{\colorbox{rouge-tres-clair}{P3}} - Sauvegarde du contexte
\end{enumerate}
\textbf{Prise en main du logiciel ArcMap de la suite ArcGis.}
\item \textbf{Tests de non-regression de la version G1R6 sur la France métropolitaine}
\begin{enumerate}[label = Tests \no\arabic*.,align=left]
\item  \emph{\colorbox{rouge}{P1}} - Impression
\end{enumerate}
\textbf{Anomalie relevé sur les zones d'égilibilités}
\newline
\textbf{Formation E-Learning}
\begin{enumerate}[label = Formation \no\arabic*.,align=left]
	\item \textit{Utiliser efficacement l'email et la messagerie isntantanee}
	\item \textit{Utiliser le Brown Paper}
	\item \textit{Utiliser du Portail MyLearning}
\end{enumerate}
\item \textbf{Tests de non-regression de la version G1R6 sur la France métropolitaine}
\begin{enumerate}[start = 2,label = Tests \no\arabic*.,align=left]
\item  \emph{\colorbox{rouge}{P1}} - Gestion FTTH - Cébles
\item  \emph{\colorbox{rouge}{P1}} - Gestion FTTH - Régles d'ingienerie
\end{enumerate}
\textbf{Formation E-Learning}
\begin{enumerate}[label = Formation \no\arabic*.,align=left]
	\item \textit{Les fondamentaux du test logiciel}
\end{enumerate}

\newpage
\item \textbf{Présentation du déroulement de mon stage} Collecte d'informations sur le centre de service  \textsc{TMA OSS\footnote{Tierce Maintenant Applicative des applications orientés réseau d'Orange}} et le domaine de compétence \textsc{SIG\footnote{Systéme d'Information Géographique}} auquel se rattache le projet Géofibre sur lequel j'effectue mon stage. Création d'un diaporama pour cette présentation. Réalisation de la présentation avec une dizaine de stagiaires, la responsable DRH et les différents chefs de projets.
\newline
\textbf{Formation E-Learning}
\begin{enumerate}[label = Formation \no\arabic*.,align=left]
	\item \textit{La politique anti-corruption du groupe}
	\item \textit{Les lois de la concurrence}
	\item \textit{Les normes écologique du groupe}
	\item \textit{Le code éthique dans la relation client}
\end{enumerate}
\item
\textbf{Correction d'anomalies hors-garantie éligibles pour la version G1R7}
\textbf{Redaction et passage des \textsc{TU\footnote{Tests Unitaires}} relatifs aux corrections}
\begin{enumerate}[label = Correction \no\arabic*.,align=left]
	\item \textit{Repositionnement d'immeubles en masse} - Perte de la sélection d'immeubles aprés avoir annulé une fenétre de choix d'immeuble.
	\item \textit{Repositionnement d'immeubles séquentiel} - Perte de la sélection d'immeubles aprés avoir annulé une fenétre de choix d'immeuble.
	\item \textit{Visu Shape} - Message d'erreur a tord "Le nombre maximum de fichiers visualisés simultanément est de 5".
\end{enumerate}
\textbf{Formation E-Learning}
\begin{enumerate}[label = Formation \no\arabic*.,align=left]
	\item \textit{Communiquer avec assurance}
	\item \textit{Entretenir de bons rapports avec le client}
\end{enumerate}

\item
\textbf{Correction d'anomalies hors-garantie éligibles pour la version G1R7}
\textbf{Redaction et passage des \textsc{TU\footnote{Tests Unitaires}} relatifs aux corrections}
\begin{enumerate}[label = Correction \no\arabic*.,align=left]
	\item \textit{Sites supports} - Perte d'information du champs gestionnaire lors de la duplication si celui-ci a la valeur "39" en production. (en attente d'informations d'Orange)
	\item \textit{Cébles, alvéoles} -  Suppression des données d'alvéoles non homogéne (en attente d'informations d'Orange)
\end{enumerate}
\textbf{Formation E-Learning}
\begin{enumerate}[label = Formation \no\arabic*.,align=left]
	\item \textit{Eléments déune équipe soudée}
	\item \textit{Etablir des relations de confiance}
	\item \textit{Etre un membre efficace au sein déune équipe}
\end{enumerate}
\textbf{Breizhcamp}
\item
\textbf{Correction d'anomalies hors-garantie éligibles pour la version G1R7}
\textbf{Redaction et passage des \textsc{TU\footnote{Tests Unitaires}} relatifs aux corrections}
\begin{enumerate}[label = Correction \no\arabic*.,align=left]
	\item \textit{Connexion} -  Geofibre ne gére pas la casse du cu\_id d'un utilisateur.
	\item \textit{Impression Libre et Casage} - Perte de la valeur par défaut du champ Résolution
\end{enumerate}
\textbf{Formation E-Learning}
\begin{enumerate}[label = Formation \no\arabic*.,align=left]
	\item \textit{Limitation des voleurs de temps}
	\item \textit{Contréler son stress}
	\item \textit{Planifier et hierarchiser son temps}
\end{enumerate}
\item
\textbf{Support é Taher qui viens d'arriver sur le projet}\\
\textbf{Correction d'anomalies hors-garantie éligibles pour la version G1R7}\\
\textbf{Redaction et passage des \textsc{TU\footnote{Tests Unitaires}} relatifs aux corrections}\\
\begin{enumerate}[label = Correction \no\arabic*.,align=left]
	\item \emph{\colorbox{rouge}{majeure}} \textit{Flux cables} -  Echec de l'import sur présence de point virgule , simple guillement ou double guillemet
\end{enumerate}
\textbf{Détection de la version d'anomalie}\\
Certaines anomalies sont en garantie (versions G1R4, G1R5, G1R6) dans quel cas si le client les trouve il faudra les corriger.
D'autres sont hors-garantie (< G1R4) Dans ce cas il faut les annoncer aux clients et ils décident si ils veulent les corriger ou non.
\\
Pour cela il faut détecter ou est-ce que l'anomalie est située dans le code et voir é quel moment les changements ont été commité sur le gestionnaire de version SVN. En fonction de la date du commit ou du TAG
ont peut remonter au numéro de version.
\begin{enumerate}[label = Détection de la version d'anomalie \no\arabic*.,align=left]
	\item \textit{Points techniques} - Il est possible créer un PT avec une référence de plus de 25 caractéres
	\item \textit{Points techniques} - Import - Probléme d'encodage dans les comptes rendus
\end{enumerate}
\textbf{Correction d'anomalies hors-garantie éligibles pour la version G1R7}\\
\textbf{Redaction et passage des \textsc{TU\footnote{Tests Unitaires}} relatifs aux corrections}
\begin{enumerate}[label = Correction \no\arabic*.,align=left]
	\item \textit{Recalcul nombre d'EL} -majBatchData.ksh,KO si  la zone est é cheval sur deux communes [ksh, SQL, PostGIS]
\end{enumerate}
\textbf{Les spécifications de la version G1R7 ont été livré et validé. Les développements peuvent commencer !}
\item
\textbf{G1R7} Lecture assidu des spécifications\\
\textbf{Correction d'anomalies hors-garantie éligibles pour la version G1R7}\\
\textbf{Redaction et passage des \textsc{TU\footnote{Tests Unitaires}} relatifs aux corrections}
\begin{enumerate}[label = Correction \no\arabic*.,align=left]
	\item \textit{Visu Shape} -sauvegarde dans le contexte utilisateur d'un shape non valide [IHM]
\end{enumerate}
\textbf{Développement G1R7 (développement, écriture et passage des tests unitaires)}
\begin{enumerate}[label = Développement \no\arabic*.,align=left]
	\item [Site support] Ajout du champs déployeur en BDD
	\item [Site support] Ajout du champs déployeur dans l'IHM
\end{enumerate}
\item
\textbf{Développement G1R7 (développement, écriture et passage des tests unitaires)}
\begin{enumerate}[label = Développement \no\arabic*.,align=left]
	\item [Annexe C3A] Nouvelle gestion du diamétre des parcours
	\item [Publication de schéma directeur] Choix Opérateur IHM autres impacts
	\item [Publication de schéma directeur] Modification des vues d'extraction
	\item [Publication de schéma directeur] Extractions filtrées sur champ Opérateur
	\item [Publication de schéma directeur] Impressions filtrées sur champ Opérateur
\end{enumerate}

\item
\textbf{Correction d'erreurs Sonar}
\\Recherche d'une solution pour remplacer le plugin Sonar pour Eclipse et Flashbuilder qui n'est plus pris en charge.
\\Mise à jour du KIT D'accueil en conséquence

\item
\textbf{Programme de Copie de données BDD} Vérification de la copie conforme des champs opérateur et déployeur.
\\ Ajout d'une contrainte liée à la configuration des RIP (si la commune d'export n'a pas
de configuration RIP alors elle prend pour valeur la configuration RIP de la table source)
\item
\textbf{Rédaction de tests d'intégration}
\begin{enumerate}[label = Développement \no\arabic*.,align=left]
	\item [Gestion des PT]  IHM - ajout/Gestion du champ opérateur + autres impacts
	\item [Gestion des PT]  BDD - Gestion nouveau champ + clé primaire
	\item [Evolution des profils] Sous-traitants - gestion des SD (gestion de configuration DSM)
	\item [Changement d’identification] Nouveaux appuis ERDF  + impacts IHM
\end{enumerate}
\textbf{Test de l'outil de migration}
\item \textbf{Rework de développement G1R7}
\\\textbf{Rework de rédaction de tests d'intégration G1R7}
\\\textbf{Tests d'intégration de la version G1R7}
\begin{enumerate}[label = Tests \no\arabic*.,align=left]
\item \emph{\colorbox{rouge}{P1}} Annexe D8
\item \emph{\colorbox{rouge}{P1}} Flux IPON - Câbles
\end{enumerate}

\item
\textbf{Tests d'intégration de la version G1R7}
\begin{enumerate}[label = Tests \no\arabic*.,align=left]
\item \emph{\colorbox{rouge}{P1}} [Parcours] Gestion du champ Opérateur.
\end{enumerate}
\textbf{Tests de l'outils de Migration}
\item
\textbf{Tests d'intégration de la version G1R7 dans les DOM}
\begin{enumerate}[label = Tests \no\arabic*.,align=left]
\item [Guyane] Câbles - Gestion du champ Opérateur.
\item [Guyane] Flux IPON - Câbles.
\item [Guadeloupe] Flux IPON - PT.
\item [Guadeloupe] Symbologie des immeubles RIP.
\item [Martinique] Filtres sur le champ Opérateur.
\item [Réunion] Evolutions des profils Sous-Traitant.
\end{enumerate}
\textbf{\colorbox{rouge}{Fin du stage. }}\\\\
\end{enumerate}

  \chapter{Shéma d'architecture technique}
\label{archtech}

\begin{figure}[h]
  \captionbox{Shéma d'architecture technique\label{fig:dummy}}{
    \includegraphics[width=16cm]{images/archtech.png}
  }
\end{figure}

\end{document}
